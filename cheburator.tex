\documentclass{article} 
\usepackage{amssymb, amsfonts, amsmath, amsthm}
\usepackage{epsfig,lpic,wrapfig}
\usepackage[T2A,T1]{fontenc}
\usepackage[utf8]{inputenc}
\usepackage[english,russian]{babel}
\usepackage{graphicx}
\usepackage{url}
%\usepackage[top=0.9in, bottom=0.9in,left=0.9in, right=0.9in, paperwidth=6in, paperheight=9in]{geometry}
\usepackage[colorlinks=true,
citecolor=black,
linkcolor=black,
anchorcolor=black,
filecolor=black,
menucolor=black,
urlcolor=black%
]{hyperref}
\hypersetup{pdftitle={Случайные блуждания.},
pdfauthor={Александр Гиль}}



\begin{document}
\title{Случайные блуждания}
\author{Александр Гиль}
\date{}
\maketitle
\begin{abstract}
Заметка основана на лекции прочитанной автором в ??? 
\end{abstract}

\section{Вероятность и математические ожидание}

Подкидывая игральную кость (кубик),
с равными шансами мы можем получить 1, 2, 3, 4, 5 или 6 очков.
Результат такого \emph{испытания} называется \emph{случайной величиной}.
Естественно предположить что этот результат не зависит от других таких же испытаний.

При увеличении числа испытаний до бесконечности,
доля испытаний на которое приходится определённый исход 
от общего их общего числа 
стремится к пределу, 
который называется его \emph{вероятностью}.
Поскольку шансы любого из шести исходов равны,
вероятность каждого исхода равна $\tfrac16$.

Представим себе, что мы увеличиваем число испытаний до бесконечности
и после каждого броска считаем среднее арифметическое полученных случайных величин.
Если эта последовательность стремится к определянному числу,
то такое число называется \emph{математическим ожиданием} или \emph{средним} случайной величины.

В нашем примере искомое среднее числа очков можно посчитать по формуле
\[1\cdot\tfrac16+2\cdot\tfrac16+3\cdot\tfrac16+4\cdot\tfrac16+5\cdot\tfrac16+6\cdot\tfrac16=3\tfrac12.\]
Действительно, в пределе, 
на долю каждого исхода 1, 2, 3, 4, 5 и 6
приходится $\tfrac16$ числа всех исходов
и значит их среднее арифметическое должно 
стремиться к левой стороне равенства.

Иначе говоря, для вычисления среднего мы должны вычислить \emph{взвешенную сумму} значений нашей величины для каждого исхода
(1, 2, 3, 4, 5 и 6), 
взяв вероятность каждого исхода ($\tfrac16$)
в качестве веса.

В этой заметке мы обсудим способы нахождения 
вероятностей 
и средних, 
в более сложной ситуации.
 
\section{Санкт-Петербургский парадокс}

Вообще говоря, может оказаться, что среднее (математическое ожидание), вычисляемое как предел для продлеваемой бесконечно серии испытаний, не существует для данной схемы испытаний.
Такое может случится поскольку у бесконечной последовательности чисел предельного значения может не существовать. 

В наших задачах такого происходить не будет,
но мы не будем это строго доказывать.

Популярным примером такой ситуации является так называемый «Санкт-Петербургский парадокс», в котором рассматривается азартная игра с повторяющимся подкидыванием монетки. 
Игра кончается, как только при очередном подкидывании выпадает орёл, и игрок получает выигрыш в размере $2^n$ дукатов 
(где $n$ --- номер первого подкидывания с орлом).
То есть, он получает $2$ дукатa если орёл выпадет при первом подкидывании; 
$4=2^2$, если при первом подкидывании выпадет решка, но при втором выпадет орёл; 
$8=2^3$ дуката, 
если при первом и втором подкидывании выпадет решка, 
но при третьем выпадет орёл, 
и так далее. 

Какую максимальную цену следует платить игроку за раунд участия в этой игре? 

Для ответа на такой вопрос вычисляют средний выигрыш; 
в данном случае это сумма следующего ряда.
В этом ряду слагаемое под номером $n$ 
--- это вероятность окончания раунда на $n$-ом подкидывании монеты, умноженная на соответствующий выигрыш.
\[\tfrac12\cdot2+(\tfrac12)^2\cdot2^2+(\tfrac12)^3\cdot2^3+\dots=1+1+1+\dots=\infty.\]
То есть средний выйгрыш в такую игру равен бесконечности.
Это приводит нас к парадоксальному выводу: 
\emph{игроку всегда выгодно покупать право на участие в этой игре, какой бы дорогой ни была цена!}

Заметим, что практический вывод о том, что игроку следует платить любую сколь угодно дорогую цену за билет на такую игру сомнителен из-за ограничений реального мира.
Например если имущество вашего соперника состоит из 1000 дукатов,
то после 9 решек он обанкротится.
В этом случае ясно, что в игре может быть максимум 8 подкидываней
и средний выйрыш равен 
\[\tfrac12\cdot2+(\tfrac12)^2\cdot4+\dots+(\tfrac12)^8\cdot 2^8=8.\]
То есть за билет на игру не следует платить больше 8 дукатов.


\section{Робот Чебуратор на коротком столе} 

Электромеханическая игрушка «Робот Чебуратор», 
будучи включенной, через равные интервалы времени делает шаги одинаковой длины налево или направо, с одинаковой вероятностью $\tfrac12$ выбирая между этими двумя направлениями. 

Предположим Чебуратор стоит на коротком столе,
шагая вправо он чебурахнется со стола,
он может шагнуть влево и остаться на столе но он чебурахнется после второго шега налево.
Нас интересуют двe zadaci:
\begin{itemize}
\item Какова вероятность того, что он чебурахнется с правого края стола, и какова --- что с левого?
\item Чему равно среднее число шагов которые сделает робот чтобы чебуранутся?
\end{itemize}
При этом мы полагаем, что существование требуемых чисел в обоих вопросах обосновано
(это действительно верно, хотя требует доказательства).
Поэтому нам остаётся лишь найти
численное значение этих чисел. 

\medskip
\noindent\textit{Решение первой задачи.}
Пусть $p_1$ и $p_2$ обозначают вероятность чебураханья направо из первого и второго положения. 
Заметим, что
\[p_1=\tfrac12\cdot0+\tfrac12\cdot p_2.\]
Действительно, после первого шага 
с вероятностью $\tfrac12$ робот чебурахнется налево,
и значит шансов черурахнутся направо у него уже небудет, 
отсюда слагаемое $0=\tfrac12\cdot0$,
и с той же вероятностью он перейдёт на второе место,
откуда у него будет вероятность $p_2$ чебурахнутся вправо,
отсюда слагаемое $\tfrac12\cdot p_2$.

Аналогично получаем уравнение 
\[p_2=\tfrac12\cdot p_1+\tfrac12\cdot 1.\]
Эти два уравнения говорят что в последовательности из четырёх числел, 
\[0,\  p_1,\  p_2,\ 1,\]
каждое является средним арифметическим своих соседей.
Иначе говоря эта последовательность является арифметической, 
и значит $p_1=\tfrac13$ и $p_2=\tfrac23$.

Заметим, что если $q_1$ и $q_2$ вероятности чебурананья налево, 
то повторув те же вычисления получаем $q_1=\tfrac23$ и $q_2=\tfrac13$.
Этого же можно добится посмотрев на задачу через зеркало, оно меняет местами право и лево.

В частности из положения 1, робот чебурахается направо и налево с вероятностями $\tfrac13$ и $\tfrac23$, 
а вероятность того что он не чебуранется равна нулю $0=1-\tfrac13-\tfrac23$.
\qed
\medskip

Аналогично решается и вторая задача.

\medskip
\noindent\textit{Решение второй задачи.}
Обозначим через $s_1$ и $s_2$ среднее число шагов которые сделает робот чтобы чебуранутся
если начинает с первой или второй позиции соответственно.
Из зеркальной симметрии (она меняет местами право и лево) следует, что $s_1=s_2$,
но мы и так это увидим скоро.

Заметим, что 
\[s_1=\tfrac12\cdot1+\tfrac12\cdot (1+s_2).\]
Действительно, после первого шага с первой клетки
с вероятностью $\tfrac12$ робот чебурахнется налево то есть он проделает всего один шаг
отсюда слагаемое $\tfrac12\cdot1$,
и с той же вероятностью он перейдёт на второе место,
откуда в среднем он пройдёт ещё $s_2$ шагов
учтя первый шаг, получаем второе слагаемое $\tfrac12\cdot (1+s_2)$.

Аналогично получаем равенство
\[s_2=\tfrac12\cdot (1+s_1)+\tfrac12\cdot 1.\]
Получаем систему из двух равенств с двумя неизвестными $s_1$ и $s_2$.
Решив эту систему из двух уравнений получаем 
$s_1=s_2=2$.
\qed
\medskip

Попробуйте решить следующую задачу тем же способом.

\medskip
\noindent\textbf{Задача.}
На рисунке вы видите план города в котором отмечен дом доктора
Шнобеля и пивная.

\ 

НУЖНА КАРТИНКА

\ 

Выйдя из дома на прогулку доктор выбирает с равной вероятности 
одну из дорог и идёт до следующего перекрёстка ровно одну минуту.
Дойдя до перекрёстка он опять выбирает с равной вероятности 
одну из дорог, возможно ту по которой он пришёл, 
и идёт до следующего перекрёстка.

Если на перекёстке есть рюмочная, 
он выпивает 50 грамм водки и продолжает свой путь.
Прогулка заканчивается как только он приходит к своему дому.

Какова средняя продолжительность прогулки доктора Шнобеля?

Сколько грамм водки выпивает доктор Шнобель в среднем за одну прогулку? 

\section{Решения прямым подсчётом}

Две задачи, рассмотренные выше, допускают решения через прямой подсчёт вероятностей. 
Мы опишем их, используя те же обозначения, что и раньше.

\medskip
\noindent\textit{Решения.}
Чтобы решить вторую задачу, 
заметим что робот чебурахается на $n$-ом шагу либо влево либо вправо с вероятностью $(\tfrac12)^n$.
А значит среднее число шагов можно найти просуммировав ряда
\[s_1=s_2=\tfrac12\cdot1+(\tfrac12)^2\cdot 2+(\tfrac12)^3\cdot 3+\dots.\]
Чтобы суммировать такой ряд нужно небольшое умение, 
но его сумма равне $2$ и не удивительно, что то же значение получено выше.

Далее заметим, что если мы начали с первой клетки, 
то робот может чебуранустся направо только на чётных шагах 
и если мы со второй клетки, 
то только не нечётных.
То есть
\begin{align*}
p_1=(\tfrac12)^2+(\tfrac12)^4+(\tfrac12)^6+\dots
\\
p_2=(\tfrac12)^1+(\tfrac12)^3+(\tfrac12)^5+\dots
\end{align*}
Применив формулу для суммы геометрической прогресси получаем те же результаты $p_1=\tfrac13$ и $p_2=\tfrac23$.
\qed
\medskip

Как видите, этот способ оказался сложней.
Кроме того, он не допускает лёгкого обобщения на случай когда стол имеет больше чем две клетки.
Как мы увидим ниже, 
в этом случае наше решение остаётся практически без изменений.


\section{Робот Чебуратор на длинном столе} 

Представим теперь, что Чебуратор стоит на длинном столе,
он чебурахнется налево ему надо сделать $i$ шагов
а чтобы чебуранутся налево ему надо сделать $m-i$ шагов.
И нас интересуют всё те же задачи.

\begin{itemize}
\item Какова вероятность того, что он чебурахнется с правого края стола, и какова --- что с левого?
\item Чему равно среднее число шагов которые сделает робот чтобы чебуранутся?
\end{itemize}

\medskip
\noindent\textit{Решение первой задачи.}
Пронумеруем возможные положения чебуратора от числами от $1$ до $m-1$,
по числу шагов налево которое чебуратору нужно сделать чтобы чебурахнутся.

Нам будет удобно добавить ещё два положения под номерами $0$ и $m$;
первое соответсвует тому что робот чебурахнулся налево 
а второй соответсвует тому что робот чебурахнулся направо
(из этих позиций выхода нет).


Обозначим через $p_i$ вероятность того, 
что Чебуратор чебурахнется направо,
если начинает с позиции под номером $i$.
Естественно мы имеем 
\[p_0=0\ \  \text{и}\ \  p_{m}=1.\]

Применим тот же метод что и раньше.
Стоя на клетке номер $i$,
с вероятностью $\tfrac12$ Чебуратор перейдёт на клетку немер $i+1$
в этом случае вероятность того, 
что он чебурахнется вправо будет $p_{i+1}$
и с той же вероятностью $\tfrac12$ он перейдёт на клетку немер $i-1$
в этом случае вероятность того, что он чебурахнется вправо будет $p_{i-1}$.
То есть 
\[p_i=\tfrac12\cdot p_{i-1}+\tfrac12\cdot p_{i+1}.\]

Иначе говоря в последовательности из $m+1$ числа
\[0=p_0,\ p_1,\ p_2,\dots,\ p_{i+j}=1\] 
каждое число является средним арифметическим 
соседей.
Отсюда получаем $p_i=\tfrac im$.
\qed

\medskip
\noindent\textit{Решение второй задачи.}
Давайте использовать ту же нумерацию позиций как и в решении первой задачи.

Обозначим через $s_i$ среднее число шагов если мы начинаем с позиции под номером $i$.
Естественно предположить, что $s_0=s_{m}=0$,
ведь попав на одну из этих клеток робот уже чебурахнулся.

Попробуем как и раньше посчитать значение $s_i$ новым спасобом.
После одного шага с $i$-ой клетки,
Чебуратор окажется на клетке $i-1$ или $i+1$ с равными вероятностями $\tfrac12$.
После чего ему останется в среднем пройти $s_{i-1}$ и $s_{i+1}$ шег соответсвенно. 
Не забыв учесть уже пройденный шаг,
получаем
\[s_i=\tfrac12\cdot(s_{i-1}+1)+\tfrac12\cdot(s_{i+1}+1).\]
или 
\[s_i=1+\tfrac12\cdot(s_{i-1}+s_{i+1}).\leqno({*}{*})\]
Мы получаем $m-1$ уравнение с 
$(m-1)$-ой неизвестной
$s_1,s_2,\dots,s_{m-1}$.

Если догадаться,
что $s_i$ зависит квадратично от $i$,
то несложно найти само решение
$s_i=i\cdot(m-i)$.
\qed
\medskip

Для закрепления материала мы советуем решить следующую задачу.

Пьяница вышел из бара, расположенного в 10 кварталах от своего дома на той же улице, чтобы вернуться домой; 
чтобы сделать дорогу домой веселее, он разнообразит её на каждом перекрёстке следующим образом. 
Дойдя до перекрёстка, он подкидывает монетку --- и, если она выпадает орлом, он продолжает путь в том же направлении; если же она выпадает решкой, он разворачивается и идёт в противоположном направлении. 
Если ему случается вернуться к бару, он всегда разворачивается в сторону дома; если же он дошёл до своего дома, он завершает свой путь. 

Сколько кварталов в среднем он пройдёт на этом пути?

Чему равно среднее число возвращений к бару при этой прогуле?

(Правильные ответы: 100 и 10 соответственно.)


\section{Бактерии в пробирке}

В пробирке живут 10 бактерий 3 зелёных и 7 жёлных.
Каждую секунду происходит следующее
с равной вероятностью, одна из бактерий погибает
и тут же одна из оставшихся (взятая с равной вероятностью)
 делится на две своих точных копии.
Таким образом в конце секунды я в пробирке остаётся ровно 10 бактерий.

Какова вероятность того что через некоторое время все станут зелёными?


Предположим, что через несколько секунд 
число зелёных бактерий стало $i$.
Тогда спустя секунду их число может стать $i-1$, $i$ или $i+1$.
При этом вероятности первого и последнего исхода равны.
Действительно, исход $i-1$ означает, 
что первая бактерия оказалась зелёной а вторая жёлтой,
а в исходе $i+1$,
наоборот первая жёлтой а вторая зелёной.
Про исход $i$ можно думать как про пропускание хода.

Таким образом наша задача становится похожей на задачу про робота чебуратора, только теперь Чебуратор ходит направо и налево с равными положительными верояностями и с оставшейся вероятнотсью пропускает ход.
Соответственно ответ в задаче можно получить подставив $i=3$ и $m=10$ в задаче про Чебуратора на длинном столе.
То есть --- с вероятностью $\tfrac{3}{10}$ все станут зелёными 
и с вероятностью $\tfrac{7}{10}$ все станут зелёными.

\medskip

Эту же задачу (а значит и первую чадачу про чебуратора)
можно решить без вычислений.
Заметим, что через некоторое время все бактерии в пробирке будут потомками одной из этих десяти бактерий и у каждой на это есть равные шансы.
Поэтому в 3 из 10 случаев все станут зелёными и в 7 из 10 все станут жёлтыми.

Остаётся только обосновать утверждение, что через некоторое время все бактерии в пробирке будут потомками одной.
Для начала заметим, что с положительной вероятностью это может случится за первые 10 секунд. 
Обозначим эту вероятность $p$;
тогда $q=1-p<1$ есть вероятность того что это не призойдёт спустя первые 10 секунд. 
Легко видеть, что вероятность того, что это не произойдёт за $10\cdot n$ меньше чем $q^n$.
Поскольку $q<1$, значение $q^n$ идёт к нулю при $n\to\infty$,
отсюда результат.

\section{Броуновское движение}

Заметьте, что если Чебуратор стоит на середине стола ---
чтобы чебурахнутся ему необходимо сделать $n$ шагов вправо и столько же влево.
Как мы выяснили, 
среднее число шагов которое он делает для тогочтоб чебурахнутся 
равно $n^2$.
Это наблюдение можно использовать в обратном направлении,
повторить это испытание много раз и 
замерить среднее количество шагов до чебуранья;
взяв корень из полученного числа мы получим размер стола 
измеренный в шагах Чебуратора.
В частности, зная размер стола мы сможем измерить шаг Чебуратора.

Поведение Чебуратора на столе мало чем отличается 
от хаотическое перемещение очень малых частиц вещества под действием ударов молекул. 
Это движение было открыл Роберт Броун в начале 19-го века, 
в начале 20-го века была предъявлена модель для этого движения.
В частности удалось оценить среднее число молекул в единице объёма по параметрам Броуновского движения.
При этом оказалось достаточным наблюдений с помощью обычного микроскопа.
Так была поставлена последняя точка в утверждении атомарной теории.

Эта физическая задача сложней определения длины шага Чебуратора описанная нами чуть выше.
Тем не менее между принципиальная идея решения у этих задач одна и та же.

\section{Да, а о чём мы говорили?}

Мы рассмотрели частный случай цепи Маркова.
Чего советуем почитать...

\end{document}

\section{Сумма геометрической прогрессии и предельные вероятности}

Нам понадобится следующая формула суммирования ряда из элементов бесконечной геометрической последовательности
\[1+x+x^2+\dots=\frac1{1-x}.\leqno({*})\]
Она верна если $|x|<1$, но нам потребуется только случай $0<x<1$.

Например, 
построив первую милю моста, 
а потом половину оставшейся мили, 
а потом половину от этого, и так далее, 
и так далее, 
через бесконечное время мы достроим полный двухмильный мост! 
В этом фантазийном примере мы взяли $x=\tfrac12$ 
и в результате бесконечного суммирования получили результат $\tfrac1{1-1/2}=2$.

Другой пример, при $x=\tfrac1{10}$, 
появляется при разложении $\tfrac{10}9$ в десятичную дробь.
\[
\tfrac{10}9
=
1.111\ldots
=
1+\tfrac1{10}+\tfrac1{100}+\tfrac1{1000}+\ldots.
\]

Предположим на секунду, 
что некая теорема
говорит нам, что выражение слева в равенстве $({*})$ имеет смысл. 
Всё, что нас просят теперь --- найти значение этой суммы. 
Обозначим это искомое число $S$;
то есть
\begin{align*}
S&=1+x+x^2+\dots=
\\
&=1+x\cdot(1+x+x^2+\dots).
\end{align*}
--- обратите внимание, в скобках --- ровно та же формула, которую мы обозначили величиной $S$ .
Давайте сделаем соответствующую подстановку и решим полученное уравнение относительно $S$
\[S=1+x\cdot S.\]
Решив уравнение относительно $S$, получаем $S=\frac1{1-x}$.
То есть мы доказали равенство $({*})$.

Однако следует быть осторожными и чётко помнить о предварительных условиях --- условиях упомянутой теоремы, которая гарантирует существование $S$.
В данном примере важно помнить, 
что $0<x<1$.
Например, возьмём $x=2$.
Eстественно предположить что сумма 
\[1+2+4+8+\dots,\]
бесконечна.
Но при этом, формула $({*})$ для этой суммы даёт $\tfrac1{1-2}=-1$.

Для точного понимания формулы $({*})$
необходимо придать смысл бесконечной суммы в левой стороне 
равенства.
Для этого обычно определяют предел последовательности 
и далее определяют сумму ряда как предел частных сумм
\[S_n=1+x+x^2+\dots+x^n\]
при $n\to\infty$.

Если вы ещё не знаете всех этих определений,
не беспокойтесь, до конца школы вам придётся их узнать.
В этом случае пропустите доказательство формулы и пользуйтесь интуитицией развитой выше. 

\medskip\noindent\textbf{Доказательство формулы $({*})$.}
Заметим
\begin{align*}
(1-x)\cdot S_n&=(1-x)\cdot(1+x+x^2\dots+x^n)=
\\
&=1-x+x-x^2+x^2-\dots-x^n=
\\
&=1-x^n.
\end{align*}
(Выражение во второй строчке иногда называют \emph{телескопической суммой} --- сумма складывается, как раздвижная подзорная труба оставляя только первый и последний член.)

Поскольку $|x|<1$, имеем $x^n\to0$,
а значит 
$(1-x)\cdot S_n=1-x^n\to 1$ при $n\to \infty$.
Последнее эквивалентно $S_n\to \frac1{1-x}$ поскольку $x\ne1$.
\qed
















Чебуратор чебурахнется направо на первом шагу с вероятностью $\tfrac12$. 
В противном случае, с вероятностью тоже $\tfrac12$, 
он переместится на левый край стола и будет иметь два равновероятных варианта второго шага. 
Таким образом, на втором шаге с вероятностью $\tfrac12\cdot\tfrac12=\tfrac14$ он чебурахнется налево, 
и с такой же вероятностью  $\tfrac14$ он вернётся к исходной позиции на правом краю стола. 
На третьем шаге (находясь на правом крае стола) он чебурахнется направо с вероятностью $\tfrac18$; 
с такой же вероятностью он шагнёт на левый край. 
На четвёртом шаге он вернётся на правый край с вероятностью $\tfrac1{16}=(\tfrac14)^2$, 
после чего чебурахнется направо на пятом шаге с вероятностью 
$\tfrac1{32}=(\tfrac14)^2\cdot\tfrac12$. 
Ситуация повторяется, и мы легко можем доказать (по индукции), что Чебуратор чебурахнется направо на шаге под номером $2{\cdot}n+1$ 
с вероятностью $(\tfrac14)^n\cdot\tfrac12$. 
Эти числа образуют геометрическую прогрессию со знаменателем $\tfrac14$. Суммируя по полученной выше формуле суммы геометрической прогрессии, получим вероятность чебураханья направо и налево
\begin{align*}
P&=
\tfrac12+\tfrac12\cdot\tfrac14+\tfrac12\cdot(\tfrac14)^2+\dots,
\\
Q&=
\tfrac14+(\tfrac14)^2+(\tfrac14)^3+\dots
\end{align*}
То есть вероятность чебураханья направо
можно подсчитать сложив все члены бесконечной геометрической прогрессии 
с первым членом $\tfrac12$ и коэффициентом $\tfrac14$.
Возможно читатель знает, 
как посчитать такую сумму и что $P=\tfrac23$ и $Q=\tfrac13$.



Отвечая на второй вопрос задачи (про предельную вероятность чебураханья Чебуратора с левого края стола), мы можем провести такую же подробную цепочку вычислений. 
Всё начнётся с необходимого условия первого шага в левую сторону, вероятность которого $\tfrac12$. 
Давайте на ситуацию после этого шага (в которой Чебуратор стоит на левом краю стола) посмотрим через зеркало, 
которое меняет левое с правым. 
Поскольку вероятности шагов влево и вправо (каждая из которых $\tfrac12$) равны, то, что мы увидим в зеркало, никак не отличается от первой части задачи! 
Таким образом, мы бы повторили те же самые вычисления (но домножив результат на $\tfrac12$ --- вероятность первого шага влево), получив в ответе $\tfrac12\cdot\tfrac23=\tfrac13$.

Получив ответы на вопросы в задаче, мы можем также ответить интересный вопрос, который не был сформулирован в условии: какова предельная вероятность того, что Чебуратор никогда не упадёт со стола? 
Ответ будет таким же, как в таком вопросе про неограниченно долгую серию подкидываний монеты: 
какова вероятность того, что монета неограниченно долго будем падать попеременно орлом и решкой (нечётные броски --- орлом, чётные --- решкой)? 
Легко понять, что вероятность сохранения интересующей нас ситуации после шага $n$ равна $(\tfrac12)^n$. 
С увеличением $n$ эта величина приближается сколь угодно близко к нолю, поэтому ответ на наш вопрос о этой предельной вероятности (третей по счёту из рассматриваемых нами) --- $0$. 
Интересно, что четвёртого варианта результата неограниченно продолжаемой последовательности испытаний из нашей задачи нет 
--- Чебуратор или чебурахается с правого края стола, 
или с левого, 
или неограниченно остаётся на столе. 
Мы определили все три предельных вероятности, и их сумма ($\tfrac23+\tfrac13+0=1$) оказалась равна $1$, 
как и в ограниченных сериях испытаний! 
Так оно и должно быть, потому что этот инвариант 
(сумма вероятностей трёх разных исходов после шага $n$ равна $1$), верный на каждом шаге, остаётся верен и при \emph{предельном переходе} («при $n\to\infty$», то есть «при $n$ стремящемся к бесконечности»). 
Заметив это, мы могли бы ответить на третий вопрос, просто отняв от $1$ сумму ответов на первый и второй вопрос ($1-\tfrac23-\tfrac13=0$).




Вычисляя $p$ мы сначала определяем вероятности того, что Чебуратор чебурахнется налево на втором шаге (это $\tfrac14$) 
и того, что на втором шаге он вернётся в исходное состояние (тоже $\tfrac14$). 
Вероятность чебураханья налево из исходного состояния после второго шага --- тоже $p$ поскольку, 
если Чебуратор остался на столе после первых двух шагов, дальнейшая неограниченная последовательность шагов ничем не отличается от исходной неограниченной последовательности шагов! Таким образом,
\[
p=\tfrac14+\tfrac14\cdot p
\]
и не сложная алгебра даёт ответ $p=\tfrac13$.

Наш результат оказался таким же, как и при решении суммированием геометрической прогрессии! 
Вы заметили, что мы воспользовались тем же приёмом «если нам дано, что искомое \emph{предельное значение} существует, 
мы легко его находим, решая уравнение», с помощью которого мы легче и короче нашли сумму бесконечной геометрической прогрессии с первым членом $1$ и со знаменателем между $0$ и $1$?

















\section{Математические ожидание}

Представим себе, что для каждого испытания 
в серии одинаковых испытаний определяется некое число,
\emph{случайная величина}.
Например, количество очков, выпадающее при броске игрового кубика, 
на $6$ гранях которого нанесены пометки-«очки» точками, от одной до шести точек)
есть случайная величина,
он может принимать шесть значений 
1, 2, 3, 4, 5, 6 и если кубик --- «честный», 
то есть каждая из шести граней выпадает при броске кубика с одинаковой вероятностью равной $\tfrac16$. 

\emph{Математическое ожидание} случайной величины или просто \emph{среднее} равно пределу среднего арифметического значений случайной величины в серии испытаний, по мере удлинения серии до бесконечности. 

\emph{Математическое ожидание} случайной величины или просто \emph{среднее} возникает, 
когда для каждого испытания в серии одинаковых испытаний определяется некое число 
(это число, обусловленное результатом испытания, и есть только что упомянутая \emph{случайная величина}) --- например, количество очков, выпадающее при броске игрового кубика (или «игральной кости»), на $6$ гранях которого нанесены пометки-«очки» точками, от одной до шести точек). 
Если кубик --- «честный», 
то есть каждая из шести граней выпадает при броске кубика с одинаковой вероятностью ($\tfrac16$), для вычисления среднего количества очков мы должны вычислить взвешенную сумму значений нашей величины для каждого исхода, взяв вероятность каждого исхода в качестве веса (важно, что сумма весов равна $1$ --- мы сделаем ошибку, если не учтём какой-то исход, вероятность которого не нулевая!). 
В нашем примере все $6$ весов равны $\tfrac16$, так что искомое математическое ожидание выпавшего количества очков равно
\[1\cdot\tfrac16+2\cdot\tfrac16+3\cdot\tfrac16+4\cdot\tfrac16+5\cdot\tfrac16+6\cdot\tfrac16=3\tfrac12.\]
--- среднему арифметическому равновероятных значений нашей величины для каждого из исходов испытания. 
В более сложных случаях вероятности исходов могут быть различны, тем не менее, термин «среднее» остаётся в силе --- так как математическое ожидание (если оно существует в данной схеме испытаний) равно пределу среднего арифметического значений случайной величины в серии испытаний, по мере удлинения серии до бесконечности. 
Применяя этот подход к примеру с игровым кубиком, мы будем раз за разом подкидывать кубик, записывать выпавшее число (целое от $1$ до $6$) и вычислять среднее арифметическое всех записанных чисел. 
При увеличении количества испытаний $n$ 
количества записей каждого 
из $6$ равновероятных исходов среди всех записанных чисел 
будут всё ближе к $\tfrac{n}6$, 
и вычисленные средние нашей случайной величины будут всё ближе приближаться к теоретически вычисленному значению математического ожидания ($3\tfrac12$). 
Теперь, когда мы рассмотрели связи математического ожидания со средним арифметическим, мы для краткости будем называть математическое ожидание случайной величины \emph{средним значением} этой величины или просто \emph{средним}.



\section{Ждём трамвай}
Давайте снова рассмотрим серию подкидываний монеты --- но теперь вместо выигрыша игрока мы будем рассчитывать среднюю длину последовательностей решек, выпавших подряд. Такая последовательность ограничена со стороны своего начала или началом серии подбрасываний монетки, или орлом, выпавшим перед первой решкой последовательности. Со стороны конца она ограничена орлом, выпадение которого завершает эту последовательность решек (заметим, что конца серии нет --- чтобы получить среднее в пределе, мы рассматриваем неограниченно длинные серии подбрасываний монетки, как в Санкт-Петербургском парадоксе). Отсортируем выписанные последовательности решек по длине и посчитаем среднюю длину, взяв длины выписанных последовательностей, помноженные каждая на свой «вес» --- предельное значение доли последовательностей такой длины среди общего количества выписанных последовательностей при стремлении длины серии подкидываний к бесконечности. 
У всех последовательностей есть одна решка в начале, и (в пределе) половина последовательностей на этом и кончаются 
(поскольку при следующем подкидывании монетки после первоначальной решки с вероятностью $\tfrac12$  выпадает орёл). 
Половина из оставшихся последовательностей (то есть $\tfrac14$ от общего количества последовательностей) имеет длину 2;
$\tfrac18$ последовательностей имеет длину 3, 
$\tfrac116$ --- длину 4 
--- и так далее. 
В результате мы можем рассчитать интересующую нас среднюю длину как сумму такого бесконечного ряда (похожего на ряд из Санкт-Петербургского парадокса --- но с другими значениями «выигрыша», обеспечивающими в этот раз сходимость ряда):
\[S=(\tfrac12)\cdot 1+(\tfrac12)^2\cdot 2+(\tfrac12)^3\cdot 3+\dots\]
Ряд этот сходится благодаря тому, что показательная функция --- вес $n$-го слагаемого --- убывает значительно быстрее линейной функции (значения усредняемой случайной величины, длины последовательности решек); мы не будем сейчас строго это доказывать. 
Опираясь на постулированную сходимость ряда, найдём его сумму с помощью уравнения, прежде всего выделив из суммируемого ряда сумму геометрической прогрессии весов и воспользовавшись формулой с прошлого занятия:
\[\tfrac12+(\tfrac12)^2+(\tfrac12)^3+\dots=\frac1{1-\frac12}-1=2-1=1\]

Давайте отнимем поэлементно ряд в левой части этого равенства от правой части определения $S$ а сумму ряда (то есть число 1) от левой части определения $S$ (то есть от $S$). 
В правой части после вынесения множителя $\tfrac12$ за скобки мы узнаем  и, решив уравнение, сможем найти значение этой величины:
\begin{align*}
S
=&(\tfrac12)+
\\
+&(\tfrac12)^2+(\tfrac12)^2+
\\
+&(\tfrac12)^3+(\tfrac12)^3+(\tfrac12)^3+
\\
+&\cdots
\end{align*}
Суммируя столбцы получаем
\[S=1+\tfrac12+(\tfrac12)^2+\dots=2.\]
Итак, мы доказали, что средняя длина непрерывной последовательности решек равна двум. 

Каково среднее количества подкидываний монеты от начала подкидываний до того, как орёл выпадает в первый раз? 
Просуммировав ряд, мы придём к ответу «один», 
но нам хотелось бы выразить это число в виде формулы от вычисленной средней длины последовательности решек (нашего $S$, которое равно $2$). 

Давайте проверим нашу вероятностную интуицию. 
С одной стороны, с началом подкидываний монеты мы можем равновероятно попасть на любое место отрезка из решек, 
так что среднее времени ожидания орла должно быть $S/2$ 
(примерно как, придя на остановку автобуса в случайный момент, мы в среднем ждём половину временного интервала между последовательными рейсами этого автобусного маршрута в это время суток). 
С другой стороны, начало непрерывной последовательности решек отличается от начала серий подкидываний монеты тем, что первое подкидывание в непрерывной последовательности решек --- обязательно решка, а последовательность начиная со второго подкидывания ничем не отличаются от любого начала подкидываний, так что среднее время ожидания орла равно $S-1$. 
Обе формулы ($S/2$ и $S-1$ дают верный результат 
(1 --- вспомним, что одно из равенств при нахождении $S$ было $S-1=\tfrac12\cdot S$), 
но какое из двух интуитивных объяснений справедливо, а какое --- нет? 

Чтобы разобраться в этом вопросе, рассмотрим серию событий (таких, как подкидываний монеты или бросков игрального кубика), 
в которых вероятность интересующего нас исхода (падения монеты орлом кверху или выпадения грани с одной точкой на игральном кубике) --- не $\tfrac12$ (как при подкидывании монеты), 
а какая-то произвольная вероятность $p$
--- например, для выпадения единицы на игральном кубике $p=\tfrac16$ а на «игральном октаэдре» (они используются в некоторых настольных играх) $p=\tfrac18$. 
Эта обобщённая схема простых случайных испытаний называется «испытания Бернулли». 
Давайте найдём и решим уравнение для интересующего нас среднего $S$.
Проведя длинную серию испытаний (записанную как последовательность чисел-исходов), выпишем в столбик все непрерывные под-последовательности (ненулевой длины!), в которых интересное нам событие не происходило, отсортировав их по мере увеличения длины. 
Доля последовательностей длины $1$ в пределе будет равна $p$ потому что такова вероятность наступления интересного нам события (например, выпадения одного очка на игральном кубике) после первого элемента в выписанных последовательностях: именно это событие обрывает последовательность.
Доля всех более длинных последовательностей будет равна $1-p$; 
далее мы сделаем уже знакомые расчёты (с той разницей, что некоторые $\tfrac12$-е заменены на $p$, а некоторые --- на $1-p$):

\begin{align*}
S&=p\cdot 1+(1-p)\cdot p\cdot 2+(1-p)^2\cdot p\cdot 3+\dots=
\\
&+p\cdot(1+(1-p)\cdot 2+(1-p)^2\cdot 3+\dots)=
\\
&=p(1+(1-p)+(1-p)^2+\dots)+
\\
&\ \ +(1-p)\cdot(p\cdot 1+(1-p)\cdot p\cdot 2+(1-p)^2\cdot p\cdot 3+\dots);
\\
S&=1+(1-p)\cdot S;
\\
p\cdot S&=1;
\\
S&=\tfrac1p.
\end{align*}



В примере с кубиком ($p=\tfrac16$) 
в результате имеем $6$ --- средняя длина отрезка испытаний между наступлением события с вероятностью $p$ равна числу $\tfrac1p$.
Таким образом, проанализировав схему испытаний Бернулли, мы заново открыли одну из форм следующего закона: средний интервал времени между наступлением некоего случайного события (такого, как выпадение конкретной грани игрального кубика) есть число, обратное частоте события (среднего количества таких событий за единицу времени). 
Рассчитывая среднее ожидание события от начала испытаний (назовём его $P$), 
мы проделаем похожие вычисления (мы можем сократить их, при первой возможности выразив $P$ через $S$):
\[P=p\cdot 0+(1-p)\cdot p\cdot 1+(1-p)^2\cdot p\cdot 2+\dots.\]
\[P=(1-p)\cdot S=\frac{1-p}{p}=\tfrac1p-1=S-1.\]

Мы пришли к выводу, что правильным был второй интуитивный подход (утверждавший, что $P=S-1$), 
а не первый (утверждавший, что $P=S/2$). 
Первый подход мог бы быть верен, 
если бы мы поставили наш эксперимент по схеме «автобусной остановки» 
с фиксированным расписанием исходов испытания 
(например, $1,2,3,4,5,6,1,2,\dots$ по кругу для чисел от $1$ до $6$), 
да ещё дополнительно разрешили начинать отсчёт ожидания на дробных временных интервалах. Как мы выяснили, это не наш случай --- случайные события не ходят по расписанию, хотя и имеют среднее время ожидания! Заметим, что если бы мы были уверены во втором подходе с самого начала, мы могли бы очень быстро найти $S$ и $P$ из двух равенств: с одной стороны,
\[P=S-1\]
(отрезав обязательный первый элемент от выписанных для вычисления $S$ последовательностей, мы получаем серию последовательностей, неотличимую от серий последовательностей в ожидании первого события от начала серии испытаний); с другой же стороны,
\[P=p\cdot0+(1-p)\cdot S\]

(когда интересное событие наступает первым после начала с вероятностью $p$, 
это даёт вклад $0$ в вычисление среднего, 
а в противном случае, вероятность которого $1-p$ вклад равен $S$). 
Итого
\[P=S-1=(1-p)\cdot S; \ p\cdot S=1; S=\tfrac1p; P=\tfrac1p-1.\]

Натренировавшись на испытаниях Бернулли, посчитаем среднее в более сложной схеме --- например, в опытах с Чебуратором из первого занятия. 
На каком в среднем номере шага он чебурахается со стола? 
Мы можем быстро посчитать это число, воспользовавшись симметрией (лево / право); назовём искомое среднее $Q$. 
Заметим, что с вероятностью $\tfrac12$ Чебуратор чебурахается на первом шаге направо, а с оставшейся вероятностью $\tfrac12$ он, 
сделав один шаг (не забудем прибавить эту единицу!), 
попадает в ситуацию, абсолютно симметричную начальной:
\begin{align*}
Q&=\tfrac12\cdot1+\tfrac12\cdot(1+Q);
\\
Q&=1+\tfrac12\cdot Q;
\\
\tfrac12\cdot Q&=1;
\\
Q&=2.
\end{align*}

Это было легко! Давайте рассмотрим задачу посложнее, с двумя параметрами.















\section{Задача про лягушку} 

Лягушка сидит на одной из двух кочек, и через каждую секунду она может или перепрыгнуть на другую кочку, или остаться на месте. 
При этом с первой кочки на вторую она прыгает с вероятностью $p$ а со второй на первую --- с вероятностью $q$. 
Если мы будем следить за лягушкой достаточно долго, какую долю времени в среднем она проведёт на первой кочке, и какую --- на второй?

Обозначив первое искомое число $P$ сразу заметим, 
что второе искомое число (которое мы обозначим $Q$) 
равно $1-P$ 
(у лягушки только два места, а временем полётов в прыжках мы пренебрегаем). 
Подсчитав $Q$ другим способом, решим уравнение и найдём  $P$.

Давайте считать, что мы записываем наблюдения за положением лягушкой в каждую секунду (записываем цифру: единицу, если в данную секунду лягушка на первой кочке, или двойку, если она на второй кочке). Для удобства рассмотрения выкинем начало записи, если мы начали наблюдать за лягушкой, когда она сидела на второй кочке: дождёмся в этом случае момента, когда она в первый раз перепрыгнет на первую кочку, и вот с этой секунды начнём записывать (так что наша последовательности начинается с единицы); поскольку вести наблюдения мы будем достаточно долго, отбрасывание стартовых двоек вычисляемые средние не изменит. Теперь каждой двойке предшествует или единица (назовём такую двойку «двойкой первого типа»), или другая двойка (тогда назовём нашу двойку «двойкой второго типа»), мы посчитаем доли двоек первого и второго типов отдельно и затем сложим, чтобы получить искомое $Q$. 
Доля двоек первого типа среди всех цифр равна $p\cdot P$ --- ведь доля единиц среди всех цифр --- это $P$, а после единицы следует двойка с вероятностью $p$. 
Доля же двоек второго типа среди всех цифр равна $(1-q)\cdot Q$ --- ведь доля двоек среди всех цифр --- это $Q$, 
а за двойкой идёт опять двойка с вероятностью $1-q$. 
Получаем уравнение с неизвестными $P$ и $Q$:
\[Q=p\cdot P+(1-q)\cdot Q.\]
Подставим $Q=1-P$ и решим относительно $P$ 
(напоследок выписав формулу и для $Q$):
\begin{align*}
1-P&=p\cdot P+(1-q)\cdot (1-P);
\\
1-P&=p\cdot P+1-q-P+q\cdot P;
\\
q&=(p+q)\cdot P;
\\
P&=\frac{q}{p+q};
\\
Q=1-P=\frac{q}{p+q}.
\end{align*}

Как мы только что выяснили, предельные вероятностей нахождения нашей системы «лягушка на кочках» в одном из двух состояний полностью определяется вероятностями переходов между состояниями. Обобщённое название таких систем (в которых вероятности переходов между состояниями в конкретный момент не зависят от истории прошлых переходов) --- \emph{марковские цепи}, 
и мы только что рассмотрели пример так называемой \emph{неприводимой} марковской цепи. 
Она отличается от марковских цепей, описывающих задачи о Чебураторе и задачу про пьяницу, в других наших задачах система за конечное число шагов приходит в одно из терминальных состояний и остаётся в нём.

Напоследок рассмотрим ещё одну задачу с большим количеством состояний системы, всегда завершающей свою эволюцию за конечное число шагов (подобно системам в задачах о Чебураторе). Нам потребуется больше вычислений, но теперь это нас не испугает!














\section{Пьяница, возвращающемся домой.}



Для решения этой задачи удобно ввести переменные таким образом, чтобы для каждого перекрёстка мы бы решали более простую подзадачу. 
Пронумеруем перекрёстки их расстоянием в кварталах от бара, и обозначим $x_i$  среднюю длину пути, который пьяница, находящийся на перекрёстке $i$, проходит, прежде чем оказаться на перекрёстке $i+1$. 
По условию задачи $x_0=1$ 
(находясь у бара, пьяница всегда идёт в верном направлении, и проходит ровно один квартал до следующего перекрёстка).
Для других же значений $i$ ($0<i<n$) 
формула вычисления  имеет два слагаемых:
\begin{itemize}
\item с вероятностью $\tfrac12$ пьяница идёт в интересующем нас (верном) направлении к перекрёстку $i+1$ 
(в этом случае длина его пути равна одному кварталу);
\item с вероятностью $\tfrac12$ пьяница идёт в другом (неверном) направлении. После того, как он проделает путь в один квартал, пьянице на перекрёстке $i-1$ понадобится в среднем пройти $x_{i-1}$ кварталов обратно к перекрёстку $i$, и затем ещё $x_i$, 
чтобы добраться до перекрёстка $i+1$.
\end{itemize}
Это рассмотрение приводит нас к следующему уравнению, которое мы решим относительно $x_i$:
\begin{align*}
x_i&=\tfrac12\cdot 1+\tfrac12\cdot(1+x_{i-1}+x_i);
\\
x_i&=2+x_{i-1}.
\end{align*}

Теперь, отталкиваясь от $x_0=1$, 
мы получаем $x_1=2+x_0=3$, 
затем $x_2=2+x_1=5$  --- и так далее (то есть $x_i=2\cdot i+1$). 
Ответ задачи (обозначим его $S$) 
равен сумме всех наших $x_i$; найти формулу, выражающую  через  нам поможет формула суммы арифметической прогрессии:
\[S=1+2+\dots+(2\cdot n-1)=n^2\]

Найдя ответ на эту задачу, мы получили интересный факт для теории марковских цепей специального вида, называемых случайными блужданиями (в данном случае --- в одном измерении, на линии) --- как мы выяснили, чтобы удалиться от начала блужданий (бара в нашей задаче) на некоторое заданное расстояние, пьяница в среднем должен потратить время, пропорциональное квадрату этого расстояния. 

